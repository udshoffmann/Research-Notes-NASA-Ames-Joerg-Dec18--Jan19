\section{NASA Application Characteristics/Models}

















%%%%%%%%%%%%%%%%%%%%%%%%%%%%%%%%%%%%%%%%%%%%%%%%%%%%%%%%%%%%%%%
%%%%%%%%%%%%%%%%%%%%%%%%%%%%%%%%%%%%%%%%%%%%%%%%%%%%%%%%%%%%%%%
%%%%%%%%%%%%%%%%%%%%%%%%%%%%%%%%%%%%%%%%%%%%%%%%%%%%%%%%%%%%%%%

\subsection{Mission Planning}

\subsubsection{Time frame}

\begin{itemize}
\item Long-term (days/weeks/months): Mission control.

    Highly complex human planning. Individual teams for individual
    aspects. Highly interdependent problems (e.g. energy
    management/solar panels/trajectory planning), global constraint
    resolution in f2f meetings.
    
    Very difficult/impossible to model precisely. Lots of human
    judgement in what can/cannot be done, safety margins etc. No
    complete, global mathematical formulation (individual subsystems
    do have quite complex models, esp. power, attitude, thermal, life
    support), or even if so not reliable (considerable built-in margin
    to ensure safety).
    
    Some interest in automation (mostly for checking constraints, some
    for plan repair, less for 'de-novo' synthesizing full
    plans). Potential scalability issues (too many different spacehips
    to control) not on horizon. Main automation issue is different
    time frames see below.
    
\item Short-term immediate reaction (seconds): Emergency procedures.

    Possible emergencies that require immediate reaction essentially
    mapped out and trained for.

\item Mid-term (hours/1 day), with significant communication delays
  (one direction 2.5 min to Mars at close approach, or 22 min to Mars
  at furthest distance): Target for automation.

    Ultimate goal/desire from crews: "replace mission control" in
    answering of questions "what should we do right now?".
    
\end{itemize}


\subsubsection{Limitations of formal models}

\begin{itemize}
\item Safety margins

    Plan at abstract level, over- or under-approximation of safe
    behaviour.
    
    Can only ever be decision support/suggestion system, not fully
    automatic control.
    
    A lot of planning takes place to stay far away from anything that
    could develop into a safety-critical situation.
    
\item Critical tasks:

    The usual way time- and safety-criticality is discussed is in
    order of severity of tasks.  In decreasing order of severity:
    'loss of crew' (someone will be injured or die), 'loss of vehicle'
    (the spacecraft is going to be destroyed or damaged', 'loss of
    mission' (the actual mission objective can't be achieved but the
    spacecraft and crew will be fine', and then on to various
    'degradations', 'inefficiencies' (in time, resource use, etc.)
    
    The usual way of handling the worst safety critical tasks is that
    the crew simply drops everything and does what is needed, right
    away.  Planning as we envision it usually requires time to think.
    So situations in which replanning takes place will be somewhat
    less critical cases, or recovery after a major event when things
    are stabilized.

\item At design-time, use explainability as a tool towards refining
  the system/to eventually establish trust of mission control experts
  into the system.

    What kind of system refinement operations make sense here?

\end{itemize}


\subsubsection{Augment, not replace, mission control}

\begin{itemize}

\item Make the best of available channel.
    
\item Big question of human factors: How to communicate in an
  X-minutes delay setting? HCI research here. Simple measures like
  non-verbal communication/logs already seem to help a lot.
    
\item From planning perspective: mission control is a kind of oracle
  with delay (physical to/fro + estimated mission-control thinking
  time).
    
\item $\Rightarrow$ Interactive planning with delay? Interesting
  option. Need "interactive planning" framework first. Suggesting
  plans, accommodate plan feedback in next iteration? Iterative plan,
  question, explain framework as per Dan/Joerg AFOSR proposal,
  Section~\ref{xpp}? Human input in form of search guidance? Real-time
  aspects could also play a role here: make the best of the planning
  time available til the next round of feedback.
    
\item $\Rightarrow$ Simpler framework: incorporate feedback not into
  planning process itself, but into the plan. Temporal concurrent
  contingent planning where sensing actions model mission control
  feedback, with duration corresponding to delay?
    
\end{itemize}



\subsubsection{Mission Planning Models}

\begin{itemize}

\item Abstract models:

    May be useful to link explicitly, via predicate characterizations,
    to low-level state and behaviour (Jeremy prior work): detect
    inconsistencies, model refinement, etc.
 
\item Build on established procedures:

    Comprehensive coded responses to events etc.
    
    Highly complex, difficult to process for humans, case distinctions
    etc.
    
    Extreme end: "planning" would "just" instantiate standard
    procedures to current situation. But usually things are not this
    simple.  Procedures are very general conditional plans, so some
    amount of thought about how they apply is often needed.  Depending
    on circumstances, procedure duration and required resources change
    a lot.
    
    Therefore: planning = search over degrees of freedom within
    standard procedures?
    
\end{itemize}


















%%%%%%%%%%%%%%%%%%%%%%%%%%%%%%%%%%%%%%%%%%%%%%%%%%%%%%%%%%%%%%%
%%%%%%%%%%%%%%%%%%%%%%%%%%%%%%%%%%%%%%%%%%%%%%%%%%%%%%%%%%%%%%%
%%%%%%%%%%%%%%%%%%%%%%%%%%%%%%%%%%%%%%%%%%%%%%%%%%%%%%%%%%%%%%%

\subsection{J. UAV Planning}


\subsubsection{Abstract / Status}

\begin{itemize}

\item UAV (mobility vehicle, air taxis). initial stages; 4- year
  perspective but depends on Uber

\item planning for individual UAV; airway traffic lanes, possible
  restrictions on speed etc; online replanning in case of need; take
  care of safety ie energy/battery, and possibility for emergency
  landings

\item planning is path over waypoints (predefined, static, at airway
  intersections), autopilot flies those; only cause for failure is
  traffic

\item language: Animal by David; hierarchical + numeric
  etc. replanning within segment of hierarchical plan.

\item simulator: NASA-made (faithful and fast-time). deterministic
  (physics of flying, under any wind conditions).

\item status: initial model in Animal; Plexil plan executor connecting
  to simulator; connected to fast-time simulator, can look at
  execution.


\end{itemize}




\subsubsection{Possible questions of common interest}

\begin{itemize}

\item Could NASA simulator be used, under a strict agreement?
  exceedingly difficult, probably; contains some Boeing code.

\item possible connection in context of online decision making /
  real-time search + safety, cf Wheeler and C6 (see below)

\item battery models a la Holger Hermanns? current plan is to assume
  some worst-case bound on battery behaviour, and restrict plan to
  remain in safe zone (proximity to landing areas) given that bound

\item Link to C6 (below): what kinds of models? continous? nondet?
  risk explicit?

\item Jeremy's abstraction predicate idea, linking state variabe
  values to continuous regions via explicit predicates on continuous
  space?

\end{itemize}





\subsubsection{Discussion model vs plans for drones in Explanation-SFB C6}

\begin{itemize}

\item PDDL style planning really makes sense only for waypoint
  map. temporal planning to model speed, timed initial literals to
  model known/expected trajectories of foreign vehicles.

\item more natural: MAP with non-controlled agents whose actions are
  fixed. could invent new PDDL dialect for this ...

\item 3D/geometric model really too hard in PDDL: pos-vector +
  speed*direction-vector is easy enough, but how to check that line
  between start/end points does not pass through blocked region? not
  possible. Even if possible, awkward and ineffective.

\item For simple planning language, without timed initial literals,
  could use simpler application where we control a drone (fleet) and
  the only dynamic element are new jobs (ie no moving obstacles are
  considered); here thinking fast online is important (only) as
  thinking longer may result in business value loss.

\item Holger's MC language: seems promising to J., also synergies with
  program analysis etc. But is it nice to model? Do the choice points
  make sense for planning?

\item Start with the simple application and language.

  Could just be standard PDDL and FD, to maximize simplicity (or
  numeric, but not sure I see use for numbers; fuel/battery simple
  model maybe).

  Could be a domain-specific setting, towards DFKI autonomous driving
  or autonomous drones, where we pick a simple programming-language
  state representation and implement ML policy and checks on top of
  that.

  Eventually, move to MC language?

\end{itemize}





















%%%%%%%%%%%%%%%%%%%%%%%%%%%%%%%%%%%%%%%%%%%%%%%%%%%%%%%%%%%%%%%
%%%%%%%%%%%%%%%%%%%%%%%%%%%%%%%%%%%%%%%%%%%%%%%%%%%%%%%%%%%%%%%
%%%%%%%%%%%%%%%%%%%%%%%%%%%%%%%%%%%%%%%%%%%%%%%%%%%%%%%%%%%%%%%


\subsection{Robot Arm Planning}

\begin{itemize}

\item Temporal planning, but sequential. Currently PDDL2.1. Need
  "temporal flexibility" if things go wrong.

\item Partial contingency planning at design-time/offline:

    Action duration intervals, lower bound / average / upper bound
    
    Plan for average durations, insert contingencies "where most
    pertinent"
    
    Observation/sensing on conditions at time points
    
    "Pertinent" quantitatively in terms of degree of violation if
    deviations from average case
    
    RFF-style algorithm (see work by Teichteil-Koenigsbuch) to
    incrementally build partial policy, possibly by outer loop around
    standard temporal planner?
    
    Note: in sequential setting, time can be treated as a state
    variable
    
    Ex. early work by Dave Smith et al in a Europa context

\item Flexible dispatching/plan flexibility/robustness at
  execution-time/online:

    Suitable where degree of flexibility required is small enough to
    not require full-scale replanning
    
\item Disjunctive STNs with duration uncertainty: lighter-weight
  alternative to contingency planning, disjunctions model alternate
  ways to reach a node
    
\end{itemize}























%%%%%%%%%%%%%%%%%%%%%%%%%%%%%%%%%%%%%%%%%%%%%%%%%%%%%%%%%%%%%%%
%%%%%%%%%%%%%%%%%%%%%%%%%%%%%%%%%%%%%%%%%%%%%%%%%%%%%%%%%%%%%%%
%%%%%%%%%%%%%%%%%%%%%%%%%%%%%%%%%%%%%%%%%%%%%%%%%%%%%%%%%%%%%%%

\subsection{Levitating-Robot-with-Fans (better name?) Planning}

\begin{itemize}

\item Robot w/o legs, propelled by fans
   
\item If several robots, multi-agent path planning becomes relevant
  here? With uncertainty about the destinations of human agents?
  
\end{itemize}
