\section{XPP: Other Discussion Notes}
\label{xpp-discussion}





\subsection{Framework}

\begin{itemize}
\item Application to constraint-based planning: 

  Same framework but additional input plan length bound B
  $\Rightarrow$ translate into SAT/CP, leverage all the techniques on
  unsat cores etc.

  What if $B$ is not given as input, but instead e.g.\ we run a
  planner once? Can we still leverage unsat core techniques etc? Seems
  doubtful. The overall thing is unsolvable so we get no bound by
  satisficing planning. Actually we need a bound on $max_{\pi \mbox{
      plan for solvable goal subset}} |\pi|$ \dots\ how to compute
  such a bound?

\item Task properties? Boolean properties of task? Use these in our
  analyses just like plan properties?

  Totally relevant in planning with resources. Widen time windows a
  little bit in Rovers domain. 

  Seems crucial for HOW type of questions, with actionable changes
  becoming task properties.

  Formal handling: \plans\ changes as function of task properties. But
  straightforward when making enforced vs analyzed property sets
  explicit.

\item Discretization vs framework with numeric-valued plan properties

  Is there a way to get around need to discretize into Boolean
  properties? No ideas for this right now. 

  Discretization seems doable/Ok. Main issue is size, many props
  needed, combinatorial explostion of analysis. 

  $\Rightarrow$ How to effectively discretize, how to choose
  thresholds?

  Generalize from example instances?

  Instance-dependent: generate some example plans, glean relevant
  thresholds from there? (relats to user dialogue where user points
  out relevant/problematic properties of example plans).

\item Allow plan properties given as blackbox code supplied by user?

\item Flexibility: 

  Proxy for risk which is one source of preferences ie one
  soft/analyzed property to consider.

  Could think about making it a first-class citizen instead, but this
  would seem to basically just turn into an explicit user preference
  so it either relates to enforced plan properties or to a combination
  with user preferences.

\item Probability distribution over \plans, causality: Jeremy is
  unconvinced. The former conflicts with trade-off explicitation; the
  latter is explicitly captured in the planning state transition model
  if you're interested in it.

\item Only satisfiably plan props are of interest (in practice: filter
  unsat ones out first)

\end{itemize}




\subsection{Computation}

\begin{itemize}
\item Compute \plans\ once and then use it for everything?

\item \cool{BDDs for some of these computations? represent the set
  \plans?}

\item When using off the shelf planner to decide entailment: compile
  properties into derive predicates; could be a convenient tool /
  elegant write-up.

\item Approximation: compute \plans' subset or superset of \plans, test
  properties/dependencies against \plans'.

  Subset: counter-examples/falsify dependencies.

  Superset: sufficient criteria finding subset of valid dependencies.

  eg planning graph represents superset of \plans; mutex analysis is a
  special case of this very general framework.

\item Too many properties to analyze all dependencies, how to select
  relevant parts of PDO to be computed?

  Relates to user dialogue in interactive setting.

  Automatic property subset selection relates to property mining.

  Can we identify interesting dependencies, ie property pairs rather
  than properties, from example plans?

\end{itemize}





\subsection{XPP vs User Preferences}

\begin{itemize}
\item PDO relation to preference hierarchy? pref $p$ is stronger than
  $q$ is what it says; but importance to user is not there.

\item In iterative-planning application scenario, user preference
  incomplete is assumed: plan generator cannot generate unique most
  preferred solution with an up-front spec of preferences, else the
  iterative process would be useless.

\item Moving border between hard vs soft constraints as you go along,
  in the iterative process. makes sense both for iterative process and
  for computation.

\item There are three reasons for preference issues between user vs
  computer:

  1. different computational power, ie bounded rationality on part of
  user; eg ``why do we do task A after 4 pm?'' ie a dependency
  $\entails{\plans}{\true}{taskAafter4pm}$, or brilliant computer go
  move not understood by user.

  2. model differences (preconditions etc).

  3. optimization/preference function not fully specified up front.

  Here we don't address 2. (no explicit support for this; could be a
  future combination). 3. is naturally addressed in iterative planning
  process. 1. could be addressed, if we have the right plan properties
  to elucidate the ``causes'' behind a dependency. Interesting
  question for future work, see next subsec and
  Section~\ref{xpp:identify-causes} below.

\item Extension combination with user preferences?

  Combine/multiply/merge PDO with preference model over properties?

  Could eg deal point out inconsistent preferences

\item Ground prefs in the reality of feasible solutions, ie compute
  exact space of feasible preferred solutions. Not sure whether it
  makes sense computationally to first compute the entire PDO for this
  purpose though.

\item Or perhaps a more interactive approach where the PDO is
  navigated guided by user preferences, or vice versa user preferenes
  are navigated guided by the PDO?

\end{itemize}







\subsection{Generating Properties Relevant to User}

Mining as per proposal: start from seed set \props, compare example
plans with \prop\ vs $\neg \prop$, identify relevant differences based
on some sort of information about what's relevant to user, use
inductive learning to characterize these differenes with new plan
properties,

\begin{itemize}
\item Concrete example in oversubscription planning:

  User declared he cares about scientific objectives and energy
  consumption. Seed property objective A. In example plans with A we
  either don't have B, or don't have C, or consume a lot of energy. In
  example plans without A we sometimes have B and C and low energy
  consumption. Infer new property ``B and C and low energy''.

\item E.g. user worries about load balancing, not in \props; how to
  discover this?

  Proposal mining approach could work in principle, but load blanacing
  (variance) difficult to discover.

\item Make property mining interactive?

  Take inspiration from preference eicitation, in setting where
  analyzed properties are soft goals? Similarity is that we are
  looking for properties relevant to user preference; difference is
  that our goal is easier, we just need relevance, not a specification
  characterizing user preferences exactly.

  \cool{Example critiquing style:} Show a plan, user provides
  properties he likes/does not like, these props are included, and new
  plan is generated keeping the ``like'' while removing the ``don't
  like''?  This setup is very close to what
  \cite{fox:etal:ijcai-ws-17} proposed as a question answering method!
  That method here bcomes instead an input to our comprehensive
  analysis. 

\end{itemize}





\subsection{User Dialogues / Interfacing}

\begin{itemize}
\item Example critiquing cf above as interactive way to collect set of
  relevant properties.

  If generating an alternative plan takes too long, inform user ``I
  couldn't figure out how to do both $A$ and ``B''? Hm contains very
  little information. 

  \cool{More interesting: Suggest relaxations that would allow to do
    this! More fuel, more time, etc. $\Rightarrow$ relaxation
    operators in search space? relax completely first then accommodate
    as many constraints as possible?} Note that this relates closely
  to ``almost plans'' (to be used to find useful new plan properties
  for causal chain in WHY answer); it also relates to HOW answers, if
  the relaxations are actionable plan/task properties; also to excuses
  where basically the suggestion is an excuse. Main difference: online
  setting where instead of a complete analysis we are trying to be
  able to have suggestions in case the search for an alternate plan
  fails.

\item How to navigate the PDO:

  Select/modify subset of property dimensions considered. eg Rovers:
  xray vs visual given memory/time/uplink passed; total science vs
  distance traveled; total science vs energy; total data collected vs
  downlinked given uplink passes.

\end{itemize}










\subsection{Concrete Examples to look at}

\begin{itemize}
\item Classical planning blocksworld?

  Could be an option if we wanna go extreme on classical setting, and
  if the blocksworld stuff has phenomena inerestingf to talk a bout
  with plan properties. TBD.

\item Oversubscription rovers.

  Playground for NASA-style things but in simple basic setting.

  ``Classical'' version of oversubscription planning \ie\ classical
  planning but too many goals to achieve given amount of one consumed
  resource.

  Can go for temporal aspects where relevant, including Dan makes
  sense.

\item Crater reconstruction?

  Sun moves, rover has to take images of crater for 3D reconstruction,
  imaging depends on rover and sun location.

  Accurate formulation has complex continuous phenomena/consraints.

  Could go for a simple discrete temporal version in PDDL2.2 ie timed
  initial literals.

\item Mission planning?

  Over-subscription setting in online/mid-term scenario when something
  went wrong and not all objectives can be achieved
  anymore. Properties e.g. achieving an objective, using/not using a
  certain device, energy consumption, meeting/not meeting a deadline.

  The simplest version of this is if some task takes much longer than
  anticipated; as this happens, the ongoing decisions are: 1) keep
  going with this task (because it is important or urgent) and not do
  something else?  2) interrupt this task to do something even more
  important/urgent?  3) stop doing this task because it's not that
  important/urgent and it can be done later? Aside: some tasks take a
  very long time (days), and are interruptible.

  In this setting, the approach would at design-time address questions
  what implications a delay has, according to the model, on other
  tasks; at execution time the approach could serve as an iterative
  planning tool, enforcing delays if the consequences are ok.

  Note though: an inherent limitation here is that, the way the
  approach is formulated with plan properties being Boolean functions,
  that the set of plan properties considered would have to discretize
  the possible time points of task completion, ie the "delays" to
  consider would have to be a fixed set of constants.

\end{itemize}



