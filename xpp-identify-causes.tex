\section{XPP: Identifying Causes behind a Dependency}
\label{xpp-identify-causes}

Scenario: given dependency $\entails{\plans}{\prop}{\propq}$. User
asks ``why does this dependency hold?''. How to answer that question?



\subsection{General Notes}

Could result in user dialogue of WHY questions going deeper at
selected points.

Two different approaches:
\begin{itemize}
\item Modifying the set of Analyzed Plan Properties

  Addresses WHY question in sense of ``elaborate the causal chain
  behind this dependency'', give more fine-grained information. Which
  intermediate properties are on the causal chain leading to
  $\entails{\plans}{\prop}{\propq}$?

\item Modifying the set of Enforced Plan Properties

  Addresses HOW question in sense of ``which enforced properties do I
  need to relax to get rid of this dependency''. Identify minimal
  relaxation under which $\entails{\plans}{\prop}{\propq}$ disappears.

  Ideally the answer should be ``actionable'' (or ``variable'', as
  opposed to ``fixed'') ie we should relax only properties that we can
  actually change if we want to.

  A milder form of this is to relax \prop.

\end{itemize}

Combinations of the two could also make sense.




\subsection{Modifying the set of Analyzed Plan Properties}
\label{xpp:identify-causes:analyzed}


Modify \props\ to make explicit the ``causal chain'' between
\prop\ and \propq. 

Modification operators: canonically, adding a new property $p \in
\candprops \setminus \props$. But removing/replacing a property could
also be of interest.

Selection of modifications must be linked to planning semantics, eg
cause for $\entails{\plans}{\true}{taskAafter4pm}$ could be other
things that must be done before task A, or could be energy consumption
at different times of day. Critical path analysis /
precondition/effect analysis / planning sub-problem analysis?


Distinguish between the properties \props\ currently being analyzed,
and the entire set of candidate properties \candprops\ that could be
in \props.


\paragraph{Modification as ``abstraction refinement''?}

Navigate in lattice of equivalence relations in the PDO? Split
equivalence classes by introducing new properties making cas
distinctions within? Minimizing number of new properties needed,
splitting ``down the middle''? 

Not an immediate correspondence, but could be useful.

1. in difference to abstractions \props\ covers only a subset of all
possible plan properties, rather than being exhaustive (entire state
space). That is, within each equivalence class we only have a subset
of the equivalent properties, and also the union of equivalence
classes is less than \candprops.

2. while in abstractions we just pretend for states to be equivalent
and are free to change our pretence in any way, here we have an
underlying semantics that fixes how things relate.

What comes closest to a refinement step seems to be this: for some $p
\in \props$, replace $p$ with a set of properties conjoining $p$ with
a case distinction, eg $p \wedge q$ and $p \wedge \neg q$. generally:
select a DNF tautology $\bigvee_{i=1}^n \phi_i$ and replace $p$ with
$\{p \wedge \phi_i\}$. This introduces $n$ equivalence classes each of
which entails $p$ ie is ordered before $\equiv{\plans}{\prop}$ in the
PDO, and where the disjunction of representatives is a member of
$\equiv{\plans}{\prop}$.

There does not seem to be a need though to use only this specific kind
of refinement. Adding arbitrary $p \in \candprops \setminus \props$
can be useful, see the following example.





\paragraph{Example: navigation on a map with fuel consumption}

State variables ``connected'' (static), ``at'', ``fuel'', ``visited''
remembering the locations one has been to; actions ``move''.

There are no enforced plan properties, so \plans\ is the set of all
action sequences are applicable in the initial state.

Concrete example: line of locations $l_{-3} ... l_3$. Initially at
$l_0$. Initial fuel $f$ something with $0 < f < 9$, let's say $f = 8$
for simplicity.

Plan properties \candprops\ considered: propositional formulas $\phi$
over state variable values $x$ and end-of-plan state variable values
$END x$, where $x$ is true if true anywhere along trajectory and $END
x$ is true if true at end of trajectory.

Analyzed properties: initially $END visited(l_{-3})$, $END
visited(l_3)$, $END \neg visited(l_{-3})$, $END \neg visited(l_3)$. As
there is not enough fuel to visit both, we get the dependencies
$\entails{\plans}{END visited(l_{-3})}{END \neg visited(l_3)}$ and
$\entails{\plans}{END visited(l_3)}{END \neg visited(l_{-3})}$.

To answer a WHY question reg these dependencies, how do we modify
\props? It seems what we need to do is add conjunctions capturing the
prerequisites for visiting both $l_{-3}$ and $l_3$. 

This seems to relate to finding a set $C$ of conjunctions leading to
$h^C(I) = \infty$ ... not sure though it's really the same thing.

If we add the properties $\{at(l_i)\}$ then we get dependencies of the
form $\entails{\plans}{at(l_i)}{at(l_{i-1})}$ for $i > 0$ and
$\entails{\plans}{at(l_{i+1})}{at(l_{i})}$ for $i < 0$, as well as
$\entails{\plans}{visited(l_{-3})}{at(l_{-3})}$ and
$\entails{\plans}{visited(l_{3})}{at(l_{3})}$.

This needs to be further refined. The simplest chain of dependencies
-- chain maximally fine-grained ands each element minimally
restrictive -- I can think up (using numeric inequalities as
abbreviation for disjunction over fuel values) is: $END visited(l_{3})
\Rightarrow$ $END visited(l_{3}) \wedge at(l_{1}) \wedge fuel \leq 7
\Rightarrow$ $END visited(l_{3}) \wedge at(l_{2}) \wedge fuel \leq 6
\Rightarrow$ $at(l_{3}) \wedge fuel \leq 5 \Rightarrow$ $END
visited(l_{-3}) \longrightarrow at(l_{2}) \wedge fuel \leq 4
\Rightarrow$ $END visited(l_{-3}) \longrightarrow at(l_{1}) \wedge
fuel \leq 3 \Rightarrow$ $END visited(l_{-3}) \longrightarrow
at(l_{0}) \wedge fuel \leq 2 \Rightarrow$ $END visited(l_{-3})
\longrightarrow at(l_{-1}) \wedge fuel \leq 1 \Rightarrow$ $END
visited(l_{-3}) \longrightarrow at(l_{-2}) \wedge fuel \leq 0
\Rightarrow$ $END visited(l_{-3}) \longrightarrow fuel < 0
\Leftrightarrow$ $END visited(l_{-3}) \longrightarrow \false \Rightarrow$
$END \neg visited(l_{-3})$.

\joerg{must it really be that awkward? TBD}

%% Consider properties of the form $\{visited(l_{3}) \wedge at(l_i)
%% \wedge fuel \leq x\}$. \joerg{TBD}
%
%% Note that we're leaving the realm of
%% propositional formulas \candprops\ here, adding numeric inequalities
%% over resources into the bargain. We have the dependencies:
%% $\entails{\plans}{visited(l_{3})}{visited(l_{3}) \wedge at(l_{3})
%%   \wedge fuel \leq 5}$, $\entails{\plans}{visited(l_{3}) \wedge
%%   at(l_{3}) \wedge fuel \leq 5}{visited(l_{3}) \wedge at(l_{2}) \wedge
%%   fuel \leq 6}$, $\entails{\plans}{at(l_{2}) \wedge fuel \leq
%%   6}{at(l_{1}) \wedge fuel \leq 7}$, $\entails{\plans}{at(l_{1})
%%   \wedge fuel \leq 7}{at(l_{0}) \wedge fuel \leq 8}$. Note that we're
%% leaving the realm of propositional formulas \candprops\ here, adding
%% numeric inequalities over resources into the bargain.
%
%% Now we could add properties $\{at(l_i) \wedge fuel \leq 8-i\}$



\paragraph{How to find new properties along the causal chain?}

Relates to regression/prerequisites.

Glean the relevant props from plan? Unclear because dependeny is about
something that canNOT happen, eg object $A$ entails not object $B$; in
the above example, the fuel problem arises only when trying to achieve
both objectives.

\cool{almost-plan: tolerate conflicts in plan; find min-conflict plan
  that achieves $A$ and $B$. $\Rightarrow$ glean props true along this
  plan?  Focus on conflict points? Focus on actionable props?}






  


\subsection{Modifying the set of Enforced Plan Properties}
\label{xpp:identify-causes:enforced}

Critical path analysis in scheduling relates to identifying causes,
taking which away removes the issue. Here: the hard constraints which
cause the dependency.

$\Rightarrow$ Find minimal weakening of enforced properties under
which $\entails{\plans}{\prop}{\propq}$ disappears?

Simple method remove minimal number of enforced properties. More
complex methods could split enforced properties into case distinctions
and remove only some of those cases.

How to capture ``change init fuel to 8 is better than change init fuel
to 9''? Need input cost fn for modifications?

Close relation to BN excuses for ``A and not B is unsolvable'', check
their approach and what it does here; also some follow-up appeared
more recently? 

Relation to model abstraction a la ASU? Refining the enforced
properties until their effect ie the entailment relation in \plans\ is
reconciled with the user expectations?

















